\documentclass[]{article}
\usepackage{lmodern}
\usepackage{amssymb,amsmath}
\usepackage{ifxetex,ifluatex}
\usepackage{fixltx2e} % provides \textsubscript
\ifnum 0\ifxetex 1\fi\ifluatex 1\fi=0 % if pdftex
  \usepackage[T1]{fontenc}
  \usepackage[utf8]{inputenc}
\else % if luatex or xelatex
  \ifxetex
    \usepackage{mathspec}
  \else
    \usepackage{fontspec}
  \fi
  \defaultfontfeatures{Ligatures=TeX,Scale=MatchLowercase}
\fi
% use upquote if available, for straight quotes in verbatim environments
\IfFileExists{upquote.sty}{\usepackage{upquote}}{}
% use microtype if available
\IfFileExists{microtype.sty}{%
\usepackage{microtype}
\UseMicrotypeSet[protrusion]{basicmath} % disable protrusion for tt fonts
}{}
\usepackage[margin=1in]{geometry}
\usepackage{hyperref}
\hypersetup{unicode=true,
            pdftitle={Regression Models Project - Motor Trend Data Analysis},
            pdfauthor={Shayan (Sean) Taheri},
            pdfborder={0 0 0},
            breaklinks=true}
\urlstyle{same}  % don't use monospace font for urls
\usepackage{color}
\usepackage{fancyvrb}
\newcommand{\VerbBar}{|}
\newcommand{\VERB}{\Verb[commandchars=\\\{\}]}
\DefineVerbatimEnvironment{Highlighting}{Verbatim}{commandchars=\\\{\}}
% Add ',fontsize=\small' for more characters per line
\usepackage{framed}
\definecolor{shadecolor}{RGB}{248,248,248}
\newenvironment{Shaded}{\begin{snugshade}}{\end{snugshade}}
\newcommand{\KeywordTok}[1]{\textcolor[rgb]{0.13,0.29,0.53}{\textbf{#1}}}
\newcommand{\DataTypeTok}[1]{\textcolor[rgb]{0.13,0.29,0.53}{#1}}
\newcommand{\DecValTok}[1]{\textcolor[rgb]{0.00,0.00,0.81}{#1}}
\newcommand{\BaseNTok}[1]{\textcolor[rgb]{0.00,0.00,0.81}{#1}}
\newcommand{\FloatTok}[1]{\textcolor[rgb]{0.00,0.00,0.81}{#1}}
\newcommand{\ConstantTok}[1]{\textcolor[rgb]{0.00,0.00,0.00}{#1}}
\newcommand{\CharTok}[1]{\textcolor[rgb]{0.31,0.60,0.02}{#1}}
\newcommand{\SpecialCharTok}[1]{\textcolor[rgb]{0.00,0.00,0.00}{#1}}
\newcommand{\StringTok}[1]{\textcolor[rgb]{0.31,0.60,0.02}{#1}}
\newcommand{\VerbatimStringTok}[1]{\textcolor[rgb]{0.31,0.60,0.02}{#1}}
\newcommand{\SpecialStringTok}[1]{\textcolor[rgb]{0.31,0.60,0.02}{#1}}
\newcommand{\ImportTok}[1]{#1}
\newcommand{\CommentTok}[1]{\textcolor[rgb]{0.56,0.35,0.01}{\textit{#1}}}
\newcommand{\DocumentationTok}[1]{\textcolor[rgb]{0.56,0.35,0.01}{\textbf{\textit{#1}}}}
\newcommand{\AnnotationTok}[1]{\textcolor[rgb]{0.56,0.35,0.01}{\textbf{\textit{#1}}}}
\newcommand{\CommentVarTok}[1]{\textcolor[rgb]{0.56,0.35,0.01}{\textbf{\textit{#1}}}}
\newcommand{\OtherTok}[1]{\textcolor[rgb]{0.56,0.35,0.01}{#1}}
\newcommand{\FunctionTok}[1]{\textcolor[rgb]{0.00,0.00,0.00}{#1}}
\newcommand{\VariableTok}[1]{\textcolor[rgb]{0.00,0.00,0.00}{#1}}
\newcommand{\ControlFlowTok}[1]{\textcolor[rgb]{0.13,0.29,0.53}{\textbf{#1}}}
\newcommand{\OperatorTok}[1]{\textcolor[rgb]{0.81,0.36,0.00}{\textbf{#1}}}
\newcommand{\BuiltInTok}[1]{#1}
\newcommand{\ExtensionTok}[1]{#1}
\newcommand{\PreprocessorTok}[1]{\textcolor[rgb]{0.56,0.35,0.01}{\textit{#1}}}
\newcommand{\AttributeTok}[1]{\textcolor[rgb]{0.77,0.63,0.00}{#1}}
\newcommand{\RegionMarkerTok}[1]{#1}
\newcommand{\InformationTok}[1]{\textcolor[rgb]{0.56,0.35,0.01}{\textbf{\textit{#1}}}}
\newcommand{\WarningTok}[1]{\textcolor[rgb]{0.56,0.35,0.01}{\textbf{\textit{#1}}}}
\newcommand{\AlertTok}[1]{\textcolor[rgb]{0.94,0.16,0.16}{#1}}
\newcommand{\ErrorTok}[1]{\textcolor[rgb]{0.64,0.00,0.00}{\textbf{#1}}}
\newcommand{\NormalTok}[1]{#1}
\usepackage{graphicx,grffile}
\makeatletter
\def\maxwidth{\ifdim\Gin@nat@width>\linewidth\linewidth\else\Gin@nat@width\fi}
\def\maxheight{\ifdim\Gin@nat@height>\textheight\textheight\else\Gin@nat@height\fi}
\makeatother
% Scale images if necessary, so that they will not overflow the page
% margins by default, and it is still possible to overwrite the defaults
% using explicit options in \includegraphics[width, height, ...]{}
\setkeys{Gin}{width=\maxwidth,height=\maxheight,keepaspectratio}
\IfFileExists{parskip.sty}{%
\usepackage{parskip}
}{% else
\setlength{\parindent}{0pt}
\setlength{\parskip}{6pt plus 2pt minus 1pt}
}
\setlength{\emergencystretch}{3em}  % prevent overfull lines
\providecommand{\tightlist}{%
  \setlength{\itemsep}{0pt}\setlength{\parskip}{0pt}}
\setcounter{secnumdepth}{0}
% Redefines (sub)paragraphs to behave more like sections
\ifx\paragraph\undefined\else
\let\oldparagraph\paragraph
\renewcommand{\paragraph}[1]{\oldparagraph{#1}\mbox{}}
\fi
\ifx\subparagraph\undefined\else
\let\oldsubparagraph\subparagraph
\renewcommand{\subparagraph}[1]{\oldsubparagraph{#1}\mbox{}}
\fi

%%% Use protect on footnotes to avoid problems with footnotes in titles
\let\rmarkdownfootnote\footnote%
\def\footnote{\protect\rmarkdownfootnote}

%%% Change title format to be more compact
\usepackage{titling}

% Create subtitle command for use in maketitle
\providecommand{\subtitle}[1]{
  \posttitle{
    \begin{center}\large#1\end{center}
    }
}

\setlength{\droptitle}{-2em}

  \title{Regression Models Project - Motor Trend Data Analysis}
    \pretitle{\vspace{\droptitle}\centering\huge}
  \posttitle{\par}
    \author{Shayan (Sean) Taheri}
    \preauthor{\centering\large\emph}
  \postauthor{\par}
      \predate{\centering\large\emph}
  \postdate{\par}
    \date{June 9, 2019}


\begin{document}
\maketitle

\subsection{Executive Summary}\label{executive-summary}

The main goal of this project is analysis of the ``mtcars'' dataset.
Next, the relationship between a set of variables and miles per gallon
(MPG) is explored in detail. We extracted the data from the year of 1974
(``Motor Trend U.S. Magazine''). It comprises fuel consumption and ten
aspects of automobile design and performance measurement for 32
automobiles (1973â€``74 models). The regresion models are used along
with the exploratory data analyses. This experiment is help in exploring
how \textbf{automatic} (am = 0) and \textbf{manual} (am = 1)
transmissions features affect the \textbf{MPG} feature. The t-test
demonstrates that the performance differenc between cars with automatic
and manual transmission. It is about 7 MPG

This analysis let us know that about seven mils per gallon (MPG) more
for cars with manual transmission than those with automatic
transmission. Next, we fit several linear regression models and select
the highest Adjusted R-Squared value. Therefore, the weight given and
the portion of ¼ mile time are kept constant. Manual transmitted cars
are defined as 14.079 + (-4.141)*Weight. More MPG on average is better
than automatic transmitted cars. Accordingly, the cars that are lighter
in weight with a manual transmission and cars that are heavier in weight
with an automatic transmission will include higher MPG amounts.

\subsection{Exploratory Data Analysis}\label{exploratory-data-analysis}

Let's load the dataset under analysis, called ``mtcars''. We change some
variables from the ``numeric'' class to the ``factor'' class.

\begin{Shaded}
\begin{Highlighting}[]
\KeywordTok{library}\NormalTok{(ggplot2)}
\end{Highlighting}
\end{Shaded}

\begin{verbatim}
## Warning: package 'ggplot2' was built under R version 3.5.3
\end{verbatim}

\begin{Shaded}
\begin{Highlighting}[]
\KeywordTok{data}\NormalTok{(mtcars)}
\NormalTok{mtcars[}\DecValTok{1}\OperatorTok{:}\DecValTok{3}\NormalTok{, ] }\CommentTok{# Sample Data}
\end{Highlighting}
\end{Shaded}

\begin{verbatim}
##                mpg cyl disp  hp drat    wt  qsec vs am gear carb
## Mazda RX4     21.0   6  160 110 3.90 2.620 16.46  0  1    4    4
## Mazda RX4 Wag 21.0   6  160 110 3.90 2.875 17.02  0  1    4    4
## Datsun 710    22.8   4  108  93 3.85 2.320 18.61  1  1    4    1
\end{verbatim}

\begin{Shaded}
\begin{Highlighting}[]
\KeywordTok{dim}\NormalTok{(mtcars)}
\end{Highlighting}
\end{Shaded}

\begin{verbatim}
## [1] 32 11
\end{verbatim}

\begin{Shaded}
\begin{Highlighting}[]
\NormalTok{mtcars}\OperatorTok{$}\NormalTok{cyl <-}\StringTok{ }\KeywordTok{as.factor}\NormalTok{(mtcars}\OperatorTok{$}\NormalTok{cyl)}
\NormalTok{mtcars}\OperatorTok{$}\NormalTok{vs <-}\StringTok{ }\KeywordTok{as.factor}\NormalTok{(mtcars}\OperatorTok{$}\NormalTok{vs)}
\NormalTok{mtcars}\OperatorTok{$}\NormalTok{am <-}\StringTok{ }\KeywordTok{factor}\NormalTok{(mtcars}\OperatorTok{$}\NormalTok{am)}
\NormalTok{mtcars}\OperatorTok{$}\NormalTok{gear <-}\StringTok{ }\KeywordTok{factor}\NormalTok{(mtcars}\OperatorTok{$}\NormalTok{gear)}
\NormalTok{mtcars}\OperatorTok{$}\NormalTok{carb <-}\StringTok{ }\KeywordTok{factor}\NormalTok{(mtcars}\OperatorTok{$}\NormalTok{carb)}
\KeywordTok{attach}\NormalTok{(mtcars)}
\end{Highlighting}
\end{Shaded}

\begin{verbatim}
## The following object is masked from package:ggplot2:
## 
##     mpg
\end{verbatim}

Afterwards, we perform preliminary exploratory data analysis. Take a
look at the \textbf{Appendix: Figures} section for the plots.According
to the box plot, it is seen that the manual transmission yields higher
values of MPG in general. From the pair of graph, it is seen zthat
higher amount of correlations among the variables such as ``wt'',
``disp'', ``cyl'' and ``hp'' exist.

\subsection{Inference}\label{inference}

In the step of inference, we make the null hypothesis of having the MPG
of the automatic and manual transmission stems from the same population
(with the assumption of the MPG that has a normal distribution). For the
purpose of graphical analysis, we use the two sample T-test.

\begin{Shaded}
\begin{Highlighting}[]
\NormalTok{result <-}\StringTok{ }\KeywordTok{t.test}\NormalTok{(mpg }\OperatorTok{~}\StringTok{ }\NormalTok{am)}
\NormalTok{result}\OperatorTok{$}\NormalTok{p.value}
\end{Highlighting}
\end{Shaded}

\begin{verbatim}
## [1] 0.001373638
\end{verbatim}

\begin{Shaded}
\begin{Highlighting}[]
\NormalTok{result}\OperatorTok{$}\NormalTok{estimate}
\end{Highlighting}
\end{Shaded}

\begin{verbatim}
## mean in group 0 mean in group 1 
##        17.14737        24.39231
\end{verbatim}

Due to having the p-value equalling to 0.00137, the null hypothesis is
rejected. Therefore, the automatic and manual transmissions are
extracted from different populations. After completion of this step, the
mean for MPG of manual transmitted cars is about seven more than that of
automatic transmitted cars.

\subsection{Regression Analysis}\label{regression-analysis}

For regression analysis step, a fit to the full model is skethed
according to the following:

\begin{Shaded}
\begin{Highlighting}[]
\NormalTok{fullModel <-}\StringTok{ }\KeywordTok{lm}\NormalTok{(mpg }\OperatorTok{~}\StringTok{ }\NormalTok{., }\DataTypeTok{data=}\NormalTok{mtcars)}
\KeywordTok{summary}\NormalTok{(fullModel) }\CommentTok{# results hidden}
\end{Highlighting}
\end{Shaded}

This model has the Residual Standard Error equalling to 2.833 on 15
degrees of freedom. Next, we calculate the adjusted R-squared value
equal to 0.779. This means the model can explain about 78\% of the
variance of the MPG variable. This is not true always since none of
coefficients are significant at 0.05 level of significance. Next, we use
backward selection for selection of some statistically significant
variables.

\begin{Shaded}
\begin{Highlighting}[]
\NormalTok{stepModel <-}\StringTok{ }\KeywordTok{step}\NormalTok{(fullModel, }\DataTypeTok{k=}\KeywordTok{log}\NormalTok{(}\KeywordTok{nrow}\NormalTok{(mtcars)))}
\KeywordTok{summary}\NormalTok{(stepModel) }\CommentTok{# results hidden}
\end{Highlighting}
\end{Shaded}

This model is written as ``mpg \textasciitilde{} wt + qsec + am''. From
this model, the residual standard error of 2.459 on 28 degrees of
freedom is achieved. Besides that, the Adjusted R-squared value is
0.8336. This means that the model can explain about 83\% of the variance
of the MPG variable. All of the coefficients are significant at 0.05
level of significance. For more information, we can refer to the
\textbf{Appendix: Figures} section for the plots again. According to the
scatter plot, it is indicated that an intersection term between ``wt''
variable and ``am'' variable exists. Due to the higher level of weight
in automatic cars than their manual counterparts. Accordingly, the
following model is developed with the inclusion of the interaction term:

\begin{Shaded}
\begin{Highlighting}[]
\NormalTok{amIntWtModel<-}\KeywordTok{lm}\NormalTok{(mpg }\OperatorTok{~}\StringTok{ }\NormalTok{wt }\OperatorTok{+}\StringTok{ }\NormalTok{qsec }\OperatorTok{+}\StringTok{ }\NormalTok{am }\OperatorTok{+}\StringTok{ }\NormalTok{wt}\OperatorTok{:}\NormalTok{am, }\DataTypeTok{data=}\NormalTok{mtcars)}
\KeywordTok{summary}\NormalTok{(amIntWtModel) }\CommentTok{# results hidden}
\end{Highlighting}
\end{Shaded}

This model has the Residual Standrad Error equalling to 2.084 on 27
degrees of freedom. The adjusted R-squared value is equal to 0.8804 that
provides us with an interpretation of the model that explains 88\% of
the variance of the MPG variable. All of the coefficients are
significant at the 0.05 level of significance. Completion of this step
leads a simple model with MPG as the outcome variable and Transmission
as the predictor variable.

\begin{Shaded}
\begin{Highlighting}[]
\NormalTok{amModel<-}\KeywordTok{lm}\NormalTok{(mpg }\OperatorTok{~}\StringTok{ }\NormalTok{am, }\DataTypeTok{data=}\NormalTok{mtcars)}
\KeywordTok{summary}\NormalTok{(amModel) }\CommentTok{# results hidden}
\end{Highlighting}
\end{Shaded}

On average, a car has 17.147 MPG using automatic transmission and 7.245
MPG with manual transmission. The model has the Residual Standard Error
as 4.902 on 30 degrees of freedom. Besides that, the Adjusted R-squared
value is equal to 0.3385. This means that the model is capable of
explaining 34\% of the variance of the MPG variable. The low Adjusted
R-squared value shows the need to add other variables to the model.
After this step, the model is complete and ready to be chosen.

\begin{Shaded}
\begin{Highlighting}[]
\KeywordTok{anova}\NormalTok{(amModel, stepModel, fullModel, amIntWtModel) }
\KeywordTok{confint}\NormalTok{(amIntWtModel) }\CommentTok{# results hidden}
\end{Highlighting}
\end{Shaded}

The model with the highest Adjusted R-squared value is selected that is
presented as: ``mpg \textasciitilde{} wt + qsec + am + wt:am''.

\begin{Shaded}
\begin{Highlighting}[]
\KeywordTok{summary}\NormalTok{(amIntWtModel)}\OperatorTok{$}\NormalTok{coef}
\end{Highlighting}
\end{Shaded}

\begin{verbatim}
##              Estimate Std. Error   t value     Pr(>|t|)
## (Intercept)  9.723053  5.8990407  1.648243 0.1108925394
## wt          -2.936531  0.6660253 -4.409038 0.0001488947
## qsec         1.016974  0.2520152  4.035366 0.0004030165
## am1         14.079428  3.4352512  4.098515 0.0003408693
## wt:am1      -4.141376  1.1968119 -3.460340 0.0018085763
\end{verbatim}

That is, a manual transmitted car that weighs 2000 lbs have 5.797 more
MPG than an automatic transmitted car that has both the same weight and
1/4 mile time. Based on this analysis, the result shows the importance
of ``wt'' (weight lb/1000) and ``qsec'' (1/4 mile time) when they stay
constant. Cars with the manual transmission has an additional level of
14.079 + (-4.141)*wt more MPG (miles per gallon) on average than cars
with automatic transmission. A manual transmitted car with the weight of
2000 lbs has 5.797 more MPG than an automatic transmitted car. This
implies that both have the same weight and ¼ mile time.

\subsection{Residual Analysis and
Diagnostics}\label{residual-analysis-and-diagnostics}

The plots are shown in the \textbf{Appendix: Figures} section. The
residual plots help us to verify the following underlying assumptions:
A. The residual versus Fitted plot shows no consistent pattern,
supporting the accuracy of the independent assumption. B. The Normal Q-Q
plot shows us the normal distribution of the residuals due to the
positioning of the points close to the line. C. The Scale-Location plot
confirms that the consistent variance assumption as the point sare
randomly distributed. D. The Residuals versus Leverage argues that no
outliers are present. All the values fall well within the 0.5 bands. E.
As for the Dfbetas, the measure of how much an observation has been
affected by the estimate of a regression coefficient.

\begin{Shaded}
\begin{Highlighting}[]
\KeywordTok{sum}\NormalTok{((}\KeywordTok{abs}\NormalTok{(}\KeywordTok{dfbetas}\NormalTok{(amIntWtModel)))}\OperatorTok{>}\DecValTok{1}\NormalTok{)}
\end{Highlighting}
\end{Shaded}

\begin{verbatim}
## [1] 0
\end{verbatim}

With these assumptions, the analyses given above will meet all the basic
assumptions of linear regression.

\subsection{Appendix: Figures}\label{appendix-figures}

\begin{enumerate}
\def\labelenumi{\arabic{enumi}.}
\tightlist
\item
  Boxplot of MPG Vs. Transmission:
\end{enumerate}

\begin{Shaded}
\begin{Highlighting}[]
\KeywordTok{boxplot}\NormalTok{(mpg }\OperatorTok{~}\StringTok{ }\NormalTok{am, }\DataTypeTok{xlab=}\StringTok{"Transmission (0 = Automatic, 1 = Manual)"}\NormalTok{, }\DataTypeTok{ylab=}\StringTok{"MPG"}\NormalTok{,}
        \DataTypeTok{main=}\StringTok{"Boxplot of MPG vs. Transmission"}\NormalTok{)}
\end{Highlighting}
\end{Shaded}

\includegraphics{Test_files/figure-latex/unnamed-chunk-10-1.pdf} 2. Pair
Graph of Motor Trend Car Road Tests:

\begin{Shaded}
\begin{Highlighting}[]
\KeywordTok{pairs}\NormalTok{(mtcars, }\DataTypeTok{panel=}\NormalTok{panel.smooth, }\DataTypeTok{main=}\StringTok{"Pair Graph of Motor Trend Car Road Tests"}\NormalTok{)}
\end{Highlighting}
\end{Shaded}

\includegraphics{Test_files/figure-latex/unnamed-chunk-11-1.pdf} 3.
Scatter Plot of MPG Vs. Weight by Transmission:

\begin{Shaded}
\begin{Highlighting}[]
\KeywordTok{ggplot}\NormalTok{(mtcars, }\KeywordTok{aes}\NormalTok{(}\DataTypeTok{x=}\NormalTok{wt, }\DataTypeTok{y=}\NormalTok{mpg, }\DataTypeTok{group=}\NormalTok{am, }\DataTypeTok{color=}\NormalTok{am, }\DataTypeTok{height=}\DecValTok{3}\NormalTok{, }\DataTypeTok{width=}\DecValTok{3}\NormalTok{)) }\OperatorTok{+}\StringTok{ }\KeywordTok{geom_point}\NormalTok{() }\OperatorTok{+}\StringTok{  }
\KeywordTok{scale_colour_discrete}\NormalTok{(}\DataTypeTok{labels=}\KeywordTok{c}\NormalTok{(}\StringTok{"Automatic"}\NormalTok{, }\StringTok{"Manual"}\NormalTok{)) }\OperatorTok{+}\StringTok{ }
\KeywordTok{xlab}\NormalTok{(}\StringTok{"weight"}\NormalTok{) }\OperatorTok{+}\StringTok{ }\KeywordTok{ggtitle}\NormalTok{(}\StringTok{"Scatter Plot of MPG vs. Weight by Transmission"}\NormalTok{)}
\end{Highlighting}
\end{Shaded}

\includegraphics{Test_files/figure-latex/unnamed-chunk-12-1.pdf} 4.
Residual Plots:

\begin{Shaded}
\begin{Highlighting}[]
\KeywordTok{par}\NormalTok{(}\DataTypeTok{mfrow =} \KeywordTok{c}\NormalTok{(}\DecValTok{2}\NormalTok{, }\DecValTok{2}\NormalTok{))}
\KeywordTok{plot}\NormalTok{(amIntWtModel)}
\end{Highlighting}
\end{Shaded}

\includegraphics{Test_files/figure-latex/unnamed-chunk-13-1.pdf}


\end{document}
